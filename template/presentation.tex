%% Author: 	Aaron <azet@azet.org> Zauner
%% License:	https://creativecommons.org/licenses/by-nc-nd/4.0/

%% theme and colorscheme
\documentclass[hyperref={draft}]{beamer}
\usecolortheme[dark,accent=white]{solarized}
\beamertemplatetransparentcovered
\setbeamertemplate{navigation symbols}{}

%% footer
\setbeamertemplate{footline}[text line]{%
  \parbox{\linewidth}{\vspace*{-15pt}
		      \insertdate \hfill \inserttitle \newline
		      \insertshortauthor \hfill \insertframenumber/\inserttotalframenumber
		     }}
\setbeamertemplate{navigation symbols}{}

%% packages
\usepackage[light,math]{iwona}
\usepackage[T1]{fontenc}
\usepackage{textpos}
\usepackage{tikz}
\usepackage{mathtools}
\usepackage{appendixnumberbeamer} 

%% title
\title{There Is No Largest Prime Number}
\subtitle{With an introduction to a new proof technique}

%% author and affilliation
\author[Aaron Zauner]{Aaron Zauner\\
        \textit{azet@azet.org}\\
        \includegraphics[height=65px,width=65px]{lambda}
       }
\institute{lambda.co.at:\\Highly-Available, Scalable \& Secure Distributed Systems}

%% venue and date
\date{Venue - 01/01/1970}


%% main
\begin{document}

{
\setbeamertemplate{footline}{} 

\begin{frame}
  \titlepage
\end{frame}

}
\addtocounter{framenumber}{-1}

{
\setbeamertemplate{footline}{}

\begin{frame}
  \tableofcontents
\end{frame}

}
\addtocounter{framenumber}{-1}


\addtobeamertemplate{frametitle}{}{%
  \begin{tikzpicture}[remember picture,overlay]
    \node[anchor=north east,yshift=1pt] at (current page.north east) {
      \includegraphics[height=30px]{lambda}
    };
  \end{tikzpicture}
}


\section{Results}
\subsection{Proof of the Main Theorem}

\begin{frame}<1>
  \frametitle{There Is No Largest Prime Number}
  \framesubtitle{The proof uses \textit{reductio ad absurdum}.}

  \begin{theorem}
    There is no largest prime number.
  \end{theorem}
  \begin{proof}
    \begin{enumerate}
      % The strange way of typesetting math is to minimize font usage
      % in order to keep the file sizes of the examples small.
    \item<1-| alert@1> Suppose $p$ were the largest prime number.
    \item<2-> Let $q$ be the product of the first $p$ numbers.
    \item<3-> Then $q$\;+\,$1$ is not divisible by any of them.
    \item<1-> Thus $q$\;+\,$1$ is also prime and greater than $p$.\qedhere
    \end{enumerate}
  \end{proof}
\end{frame}

%% appendix
\appendix
\begin{frame}{Bonus Slides}
blablabla
\end{frame}


\end{document}

